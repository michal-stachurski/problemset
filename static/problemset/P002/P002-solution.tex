\documentclass[12pt,preview,border=5pt,varwidth=true]{standalone}

\usepackage[utf8]{inputenc}
\usepackage{polski}

\usepackage{geometry}
\usepackage{amsmath,amssymb,amsthm}

\begin{document}

Początkowo weźmy 6 monet i rozdzielmy na dwie szalki -- po 3 na każdą. Rozważmy dwa przypadki:
\begin{enumerate}
    \item[1)] Szalki są w równowadze. Wtedy wiemy, że w pozostałych 6 monetach jest ta fałszywa. Weźmy więc dowolne 4 z nich i rozłóżmy po 2 na każdej szalce. Znowu, jeśli monety ważą tyle samo to zostały nam dwie, wówczas bierzemy jedną z nich i ważymi z inna prawdziwą monetą.

    \item[2)] Jeśli jedna szalka przeważa, to powtóżmy procedurę z 1) jednak używają tych 6 monet którę są obecnie na wadze (te których nie ważyliśmy w pierszym ważeniu są na pewno prawdziwe). Zostały nam dwa ważenia, czyli tyle samo co w 1).
\end{enumerate}

\end{document}